%%%%%%%%%%%%%%%%%%%%%%%%%%%%%%%%%%%%%%%%%
% Beamer Presentation
% Standard LaTeX Template used for creating presentation of Firebird-V Robot and other tutorials. 
% Author: Saurav Shandilya (e-Yantra Team)
% Reference: www.LaTeXTemplates.com Version 1.0 (10/11/12)
%
%%%%%%%%%%%%%%%%%%%%%%%%%%%%%%%%%%%%%%%%%

%----------------------------------------------------------------------------------------
%	PACKAGES AND THEMES
%----------------------------------------------------------------------------------------
		
\documentclass[table,10pt,red]{beamer}	% First line -- Define document class as Beamer which is used for creating presentation using Latex
\setbeamercolor{alerted text}{fg=blue} 	% Sets color of highlighted text during presentation.  
 

% The Beamer class comes with a number of default slide themes
% which change the colors and layouts of slides. Below this is a list
% of all the themes, uncomment each in turn to see what they look like.

\usetheme{default} 
%\usetheme{AnnArbor} %N
%\usetheme{Antibes} %M
%\usetheme{Bergen} %N
%\usetheme{Berkeley} %N
%\usetheme{Berlin}		%used theme in present documents.
%\usetheme{Boadilla} %N
%\usetheme{CambridgeUS} %N
%\usetheme{Copenhagen} %N
%\usetheme{Darmstadt} %N
%\usetheme{Dresden} %Y
%\usetheme{Frankfurt} %N
%\usetheme{Goettingen} %N
%\usetheme{Hannover} %N
%\usetheme{Ilmenau} %N
%\usetheme{JuanLesPins} %Y
%\usetheme{Luebeck} %Y
%\usetheme{Madrid} %N
%\usetheme{Malmoe} %Y
%\usetheme{Marburg} %N
%\usetheme{Montpellier} %Y
%\usetheme{PaloAlto} %N
%\usetheme{Pittsburgh} %Y
%\usetheme{Rochester} %Y
%\usetheme{Singapore} %yes
%\usetheme{Szeged}
%\usetheme{Warsaw}

% As well as themes, the Beamer class has a number of color themes
% for any slide theme. Uncomment each of these in turn to see how it
% changes the colors of your current slide theme.
 
%%Szeged+dove

%\usecolortheme{albatross} %dark blue N
%\usecolortheme{beaver} %white and grey
%\usecolortheme{beetle} %blue and dark grey N
%\usecolortheme{crane} %yellow blue base: white
%\usecolortheme{dolphin} %pink N
%\usecolortheme{dove} %black and white
%\usecolortheme{fly} base: grey N
%\usecolortheme{lily}  %red white black (previously used)
%\usecolortheme{orchid} %similar to lily N
%\usecolortheme{rose}  %similar to lily N
%\usecolortheme{seagull} %grey and white. N
%\usecolortheme{seahorse}% red pink base white N
%\usecolortheme{whale} similar to lily N
%\usecolortheme{wolverine} %yellow red orange base white

%\setbeamertemplate{footline} % To remove the footer line in all slides uncomment this line
%\setbeamertemplate{footline}[page number] % To replace the footer line in all slides with a simple slide count uncomment this line

%\setbeamertemplate{navigation symbols}{} % To remove the navigation symbols from the bottom of all slides uncomment this line
%}

%------------------------------------------------------------------------------------------
%	\usepackage is required for including various features like images, table, references etc.
%	Packages must be installed before using. These can be istalled through package manager. 
%   Various packages have dependencies and for using such packages all dependent packages must be used. 
%-----------------------------------------------------------------------------------------
%\usepackage{beamerthemeshadow} % theme shadow for visual 
\usepackage{beamerthemesplit} % Creates minipage (for showing multiple images and text) on same page  
\usepackage{graphicx} % Allows including images
\usepackage{booktabs} % Allows the use of \toprule, \midrule and \bottomrule in tables
\usepackage{xcolor}
\usepackage{booktabs,array}
\usepackage{listings}
\usepackage{hyperref}	% Required for including hyperlink in document
\usepackage{verbatim,moreverb} % Required for including code snippet.
\usepackage{colortbl}
\usepackage{multirow}	% Required for creating multiple row tables
\usepackage{tikz}		% Required for drawing shapes such as circles, arrowed line, etc. 
\usetikzlibrary{arrows}

% logo
%\logo{\includegraphics[height=1cm]{iitblogo.pdf}} % includes logo at bottom of all slides 

%----------------------------------------------------------------------------------------
%	TITLE PAGE
%----------------------------------------------------------------------------------------
% sf family, bold font
\sffamily \bfseries
% content inside [] appears at bottom of all page. content inside {} appears on first page as title. double backslash means line change 
\title
[
Uncertainty and Disturbance Estimator Based Robust Pitch Autopilot	% bottom of all page
%	\hspace{0.5cm}
%	\insertframenumber/\inserttotalframenumber
]
{
	Uncertainty and Disturbance Estimator Based Robust Pitch Autopilot
}

\author
[
%RAKSHITH VISHWANATHA, SHARATH RAO, ABHISHEK BASRITHAYA, T.S. CHANDAR	%Name at bottom of all page 
]
% author name on title slide
{
	RAKSHITH VISHWANATHA, SHARATH RAO, \\ABHISHEK BASRITHAYA, T.S. CHANDAR \\
  %P.E.S. Institute of Technology,Bangalore\\
}
%\date
%{
%P.E.S. Institute of Technology, Bangalore, India \\ {\today}	%\today picks system date on title slide
%}
\date
{
P.E.S. Institute of Technology, Bangalore, India \\ 17th October 2018	%\today picks system date on title slide
}
\begin{document} % IN LATEX ALL DOCUMENT/REPORT/PRESENTATION STARTS WITH \begin{document} AND ENDS WITH \end{document}

\begin{frame}	% FRAME MEANS SLIDE. \begin{frame} STARTS THE SLIDE AND \end{frame} ENDS THE SLIDE
	\titlepage % Print the title page as the first slide
\end{frame}
\section*{Outline}
% START OF SECOND SLIDE
\begin{frame}
	\frametitle{Outline of work} % Table of contents slide, comment this block out to remove it
	\tableofcontents % Throughout your presentation, if you choose to use \section{} and \subsection{} commands, these will automatically be printed on this slide as an overview of your presentation
\end{frame}
%----------------------------------------------------------------------------------------
%	PRESENTATION SLIDES
%----------------------------------------------------------------------------------------

\section{Problem Statement} % Sections can be created in order to organize your presentation into discrete blocks, all 
%sections and subsections are automatically printed in the table of contents as an overview of the talk
\begin{frame}
	\frametitle{Problem Statement}
	%\textbf{How do actuators pose a problem in system stability?}
 	\begin{itemize}  % Shows text in bullet point 	
		\item Missile control systems have traditionally faced many challenges in predicting the uncertainty and disturbance acting on the missile at various points in its flight envelope. %Actuators cause drastic deterioration in stability unless compensation is provided 
		\medskip
		\item Few robust control techniques (SMC, PC) can tackle non-linear system, but it requires prior knowledge of the magnitude bounds of the disturbance acting on the system %One of the reasons for this instability, is lag created by the actuators
		\medskip
		\item Uncertainty and Disturbance Estimator (UDE) has been employed to address these concerns as it does not require the knowledge of uncertainty and its bounds. %Our work consists of strategies to provide compensation for this kind of actuator problem
	\end{itemize}
\end{frame}
%------------------------------------------------
%\section{Abstract}
%\begin{frame}
	%\frametitle{Abstract}
 %Design of robust autopilots along with actuator dynamics compensation continues to be a challenging task. Few works are available in the literature to address this issue. In this work an attempt has been made to design a robust roll autopilot using the technique of uncertainty and disturbance estimator, assuming ideal actuator. Further, considering a second order actuator, three methods have been proposed for actuator compensation. Numerical simulations have also been carried out to ascertain the efficacy of the proposed methods.\\		
%\end{frame}
%------------------------------------------------SECTION1-----------------------------------------------------------------------------------
\section{Missile Model} % Sections can be created in order to organize your 
\subsection{Missile Model}
	\begin{frame}
	\frametitle{Missile Model}
	 
	%\textbf{Block diagram schematic for roll dynamics of a tactical missile} 	
 	\begin{itemize}  % Shows text in bullet point 		
		\item The missile model is a pitch axis,
		longitudinal, tail controlled missile with nonlinear dynamics referred from [\ref{geso}];
		\begin{enumerate}
				\item $\dot{\alpha}(t)=	K_\alpha M(t) C_n [\alpha(t),\delta(t),M(t)]cos(\alpha(t)) + q(t) $
				\item $\dot{q}(t) =	K_q M^2(t) C_m [\alpha(t),\delta(t),M(t)]$\\
				\item $\dot{\delta}(t)	=	-\omega_a\delta(t) + \omega_a\delta_c(t)$
 		\end{enumerate}
		\end{itemize}
	\begin{itemize}
		\item  where;
			\begin{enumerate}
				\item $C_n(\alpha,\delta,M)=a_n\alpha^3+b_n\alpha|\alpha|+c_n\Big(2-\frac{M}{3}\Big)\alpha\quad+ d_n\delta$
				%\item Robots and external devices
				\item $C_m(\alpha,\delta,M)=a_m\alpha^3+b_m\alpha|\alpha|+ c_m\Big(-7+\frac{8M}{3}\Big)\alpha 
				\quad+d_m\delta$
					
				 \item $\dot{M}(t)=\frac{1}{\nu_s}[-|a_z(t)|sin|\alpha(t)|+a_xM^2(t)cos\alpha(t)]$
				 %\item $a_z = K_z M^2(t)C_n[\alpha(t),\delta(t),M(t)]$
				 %\item $a_x = \frac{0.7P_0SC_d}{m}$
			\end{enumerate}
	\end{itemize}
	\begin{itemize}
		\item Assumes constant post burnout mass, no roll rate, zero roll angle, no sideslip and no yaw rate \textcolor{red}{CAN REMOVE MASS AND ROLL ANGLE}
		\item \textcolor{red}{Its better if there is a point where we say $\delta_c$ is the control input and $\alpha$ is the parameter to be controlled}
\end{itemize}

	%---------------------------------------------------figure 1 block diagram schematic------------------------------------------------------

	\end{frame}
 	%----------------------------next slide------------------------------------------------------------------------------------------
 	
 	\begin{frame}
 	\frametitle{Aerodynamic constants}
 	\begin{itemize}
 		\item Model design constraints:
 		\begin{enumerate}
 			\item $-20^\circ\le \alpha \le20^\circ$
 			\item $1.5 \le M \le 3$
 			\item $\pm25\%$ uncertainty in $C_n$ and $C_m$
 		\end{enumerate}
 	\end{itemize}
 	\begin{table}[h]
 		\begin{center}
 			\caption{Performance specifications}\label{tb1}
 			\begin{tabular}{ccc}
 				\hline
 				$\omega_a$ & Actuator bandwidth & 150 rad/s\\ \hline
 				$\zeta_a$ & Actuator damping & 0.7 \\ \hline
 				$m$ & Mass & 204.023 kg\\ \hline
 				$d$ & Diameter & 0.2286 m \\ \hline
 				$I_y$ & Pitch moment of inertia & 247.44 $kgm^2$\\ \hline
 				$C_d$ & Drag moment & 0.3\\ \hline
 				$M$ & Mach & 2.25 \\ \hline
 				$CHECK$ & IF ALL & REQD PRESENT \\ \hline
 		
 			\end{tabular}
 		\end{center}
 	\end{table}
 \end{frame}
 	
	\begin{frame}
	\frametitle{Block Diagram of UDE Controller-Observer}
		\textbf{Change Ke to $\nu$?? AND show UDE's role in Observer?}
		\begin{figure}
			\includegraphics[width=0.8\linewidth]{block_diag}
		\end{figure}

\end{frame}


%------------------------------------------------
%SUBSECTION1
\subsection{Controller Design} % A subsection can be created just before a set of slides with a common theme to further break down your presentation into chunks
\begin{frame}
	\frametitle{Input Output Linearization}
	\textbf{Implementation of IOL to cancel non-linearities of missile model [\ref{eso}]} 
	\begin{itemize}
		\item To make relative degree = order of equation, $d_n \approx 0$
		\item Within $-20^\circ\le \alpha \le20^\circ$, $\cos(\alpha) \approx 1$
		\item Obtaining $\dddot{\alpha}$
	\end{itemize}

\begin{eqnarray*}
\begin{aligned}
	\dddot{\alpha}&=K_q M^2(3a_m\alpha^2+2b_m|\alpha|+c_m\Big(-7+\frac{8M}{3}\Big))\dot{\alpha}\\ 
	&\quad - K_q M^2d_m\omega_a\delta+K_{\alpha}M(6a_n\alpha+2b_n sgn(\alpha))\dot{\alpha}^2\\ 
	&\quad + K_\alpha M(3a_n\alpha^2+2b_n|\alpha|+c_n\Big(2-\frac{M}{3}\Big))\ddot{\alpha}\\ 
	&\quad + K_q M^2d_m\omega_a\delta_c \label{a3dot}
\end{aligned}
\label{eq4}
\end{eqnarray*}

\end{frame}

\begin{frame}
When represented in IOL form $\dddot{\alpha} = a + b\delta_c$ we get; 
\begin{eqnarray*}
	\begin{aligned}
		a &= K_qM^2(3a_m\alpha^2+2b_m|\alpha|+c_m\Big(-7+\frac{8M}{3}\Big))\dot{\alpha} \\ 
		&\quad -K_qM^2d_m\omega_a\delta+K_\alpha M(6a_n\alpha+2b_nsgn(\alpha))\dot{\alpha}^2\\ 
		&\quad+K_\alpha M(3a_n\alpha^2+2b_n|\alpha|+c_n\Big(2-\frac{M}{3}\Big))\ddot{\alpha} \\
		b &= K_q M^2d_m\omega_a \nonumber
	\end{aligned}
	\label{eq5}
\end{eqnarray*}

Thus the control law is, 

\begin{eqnarray*}
	\begin{aligned}
%		without this comment the equations on this slide disappear. No idea why.
		\delta_c &= \frac{1}{b}(u_a+\nu) \\ \label{iol_control}
		u_a &= -a \label{ua_eqn}\\
		\nu &= \dddot{\alpha}^\ast+m_1(\alpha^\ast-\alpha) + m_2(\dot{\alpha}^\ast-\dot{\alpha}) + m_3(\ddot{\alpha}^\ast-\ddot{\alpha}) \label{nu_eqn}
	\end{aligned}
	\label{eq5}
\end{eqnarray*}

\end{frame}
%-------------------------------------------------------------------------------------------------------------------------------------------
\begin{frame}
\frametitle{UDE Augmented IOL Controller}
\textbf{The UDE control law utilizes a new term $u_d$ as follows [\ref{ude1}]}
\begin{itemize}  % Shows text in bullet point 		
		\item $\delta_c = \frac{1}{b}\Big[u_a+u_d+\nu\Big]$ where, $u_d = -\hat{d}$
		\item $\hat{d}$ is an estimate of the lumped disturbance and uncertainties $d$;
%\end{itemize}

%\textbf{$\hat{d}$ is an estimate of the lumped disturbance and uncertainties $d$;}
%\begin{itemize}
		\begin{enumerate}
			\item $\dddot{\alpha} = a + b\delta_c + d$
			\item $d = \Delta a + \Delta b \delta_c + w$
			\item $\hat{d}=G_f(s)d$
			\item $G_f(s)=\frac{1}{1+s\tau}$ \\
		\end{enumerate}
\end{itemize}

\begin{itemize}
	\item Thus, we finally get
	$u_d=\frac{-1}{\tau}\Big[\ddot{\alpha}-\int{\nu dt}\Big] \label{ude}$
	
\end{itemize}

\end{frame}
%%------------------------------------------------------------------------------------------------------------------------------------------
\begin{frame}
\frametitle{UDE Observer based Control law}
	\textbf{A Luenberger like UDE Observer has been designed as given in [\ref{talole2011}]}


\begin{itemize}  % Shows text in bullet point 		
	\item $\dddot{\alpha}$ is separated into linear and non-linear ($d_1$) terms; $\dddot{\alpha} = a_1 \alpha + a_2 \dot{\alpha} + a_3 \ddot{\alpha} + d_1 + b\delta_c$
	
	\item Then the non-linear term $d_1$ and uncertainties are clubbed into a lumped term $d_2$, such that; \\
	$\dddot{\alpha} = a_{1o} \alpha + a_{2o} \dot{\alpha} + a_{3o} \ddot{\alpha} + b_o\delta_c + d_2$\\
	
	\item This expressed in state space form can then be used to design the observer:
	\begin{eqnarray*}
		\begin{aligned}
			\dot{x}_1 &= x_2 \\
			\dot{x}_2 &= x_3 \\
			\dot{x}_3 &= a_{1o}x_1 + a_{2o}x_2 + a_{3o}x_3 + b_o \delta_c + d_2 \\
			y &= x_1 \label{rx1}
		\end{aligned}
		\label{eq5}
	\end{eqnarray*}
	
\end{itemize}

\end{frame}

\begin{frame}
\frametitle{UDE Observer based Control law}

Now since the equations are represented in a linear manner, a Luenburger like UDE Observer is designed by introducing the observer poles $[\beta_1 \beta_2 \beta_3]$
\begin{eqnarray*}
	\begin{aligned}
		\dot{\hat{x}}_1 &= \hat{x}_2 + \beta_1 e_o\\
		\dot{\hat{x}}_2 &= \hat{x}_3 + \beta_2 e_o\\
		\dot{\hat{x}}_3 &= a_{1o}\hat{x}_1 + a_{2o}\hat{x}_2 + a_{3o}\hat{x}_3 + b \delta_c + \hat{d}_2 + \beta_3 e_o\\		
		\hat{y} &= \hat{x}_1 
	\end{aligned}
	\label{eq5}
\end{eqnarray*}

Here, the term $d_2$ representing the non-linearities and uncertainties is estimated by UDE

\end{frame}



\begin{frame}
\frametitle{Stability Analysis}

\textbf{Error dynamics for UDE missile autopilot controller-observer structure}
\begin{eqnarray*}
\begin{aligned}
\begin{bmatrix}
\dot{e}_c \\
\dot{e}_o \\
\dot{\tilde{d}}_2
\end{bmatrix} =& 
\begin{bmatrix}
(A - BK) & -(BK) & -B_d \\
0 & (A - LC) & B_d \\
0 & 0 & -\frac{1}{\tau}
\end{bmatrix}
\begin{bmatrix}
e_c \\
e_o \\
\tilde{d}_2
\end{bmatrix}	
& + 
\begin{bmatrix}
0 \\
0 \\
1
\end{bmatrix} \dot{d}_2 \label{sr8}
\end{aligned}
\end{eqnarray*}

Eigen values of the system matrix can be computed from
\begin{eqnarray*}
\begin{vmatrix}
sI - (A - BK)
\end{vmatrix}
\begin{vmatrix}
sI - (A - LC)
\end{vmatrix}
\begin{vmatrix}
s - (-\frac{1}{\tau})
\end{vmatrix} = 0 \label{sr9}
\end{eqnarray*}

\begin{itemize}
	\item $(A, B)$ is controllable and $(A, C)$ is observable
	\item $\tau$ is strictly a positive number
	\item Selecting appropriate controller and observer poles ensures stability of error dynamics
	\item Also if $\dot{d}_2 \ne 0$, then bounded-input, bounded-output stability can be assured. 
\end{itemize}


\end{frame}


%%--------------------------------------------- --
%% Start of fifth slide
%%------------------------------------------------------------------------------------------------------------------------------------------

\begin{frame}
\frametitle{Performance of UDE based controller}
\textbf{Parameters for simulation}
\begin{itemize}  % Shows text in bullet point 		
		\item filter constant $\tau=0.01$
		\item desired settling time $t_s= 180$ ms
		\item damping factor $\zeta = 0.8$
				\begin{enumerate}
					\item Using these values, feedback gains $m_1$ and $m_2$ were evaluated to be $m_1=42.45$, $m_2=771.13$ 
					%\item 
				\end{enumerate}
		\item external disturbance $d_{ext}= 200$ rad/$s^2$
		\item taking $\omega_{RR}$ to be -3 rad/s against the nominal value of 2 rad/s
		\item desired roll orientation $=0$ deg
		\item initial condition in $\phi= 10$ deg
		\item all other values as referred from Table.1
	\end{itemize}
	
\textbf{For this controller and plant system, the phase margin was found to be 69 deg, validating the proposed control law}
\end{frame}
%Choosing $\tau$ to be 0.01, a desired settling time of 180 ms and a damping factor of 0.8 magnitude, the feedback gains $m_1$ and $m_2$ were evaluated as 42.45 and 771.13, respectively. Referring to Table I, we assume the external disturbance of 200 rad/$s^2$ and desired roll orientation to be 0 deg. With this choice of $\tau$, simulation results while considering an initial condition of $\phi$ to be 10 deg, are shown in Fig. 2.
%%-----------------------------------------------------------------------------------------------------------------------------------------
\begin{frame}
\frametitle{Performance of UDE (Case 1)}
%\textbf{Simulation Output}
\begin{figure}
\includegraphics[width=4cm]{1_ude_varying-mach_x1}
%\title{Output response}
\includegraphics[width=4cm]{2_ude_varying-mach_control}
\end{figure}



\end{frame}	

\begin{frame}
\frametitle{Performance of UDE (Case 1)}
\begin{figure}
	\includegraphics[width=4cm]{3_ude_varying-mach_mach}
	%\title{Output response}
	\includegraphics[width=4cm]{4_ude_varying-mach_dist}
\end{figure}
	\begin{center}
	\includegraphics[width=4cm]{5_ude_varying-mach_pitch}
	\end{center}

\end{frame}
%%------------------------------------------------
%%% Start of sixth slide

\section{Performance of UDE based controller with second order actuator in the loop}
%\subsection{Without Actuator compensation}
\begin{frame}
\textbf{Cascading second order actuator}
Second order actuator as referred in \cite{talole2011} of the form ${{\delta (s)}\over{\delta_c(s)}} = {{\omega^2_A}\over{s^2 + 2\zeta_A \omega_A s + \omega^2_A}}$ is introduced
\begin{itemize}  % Shows text in bullet point 		
		\item $\omega_A$ is actuator bandwidth in rad/s 
		\item $\zeta_A$ is actuator damping ratio
		%\item damping factor $\zeta = 0.8$
				%\begin{enumerate}
					%\item Using these values, feedback gains $m_1$ and $m_2$ were evaluated to be 
					%\item $m_1=42.45$, $m_2=771.13$ 
				%\end{enumerate}
		%\item external disturbance $d_{ext}= 200$ rad/$s^2$
		%\item desired roll orientation $=0$ deg
		%\item initial condition in $\phi= 10$ deg
		%\item all other values as referred from Table.1
\end{itemize}
\begin{figure}
	\includegraphics[width=0.8\linewidth]{fig4}
	\end{figure}
\end{frame}
%-------------------------------------------
\begin{frame}
\frametitle{Performance of UDE based controller with second order actuator in the loop}
\textbf{For simulation;}
\begin{itemize}  % Shows text in bullet point 		
		\item $\tau$ continues to be 0.01
		\item All other simulation parameters are as per Table.1
		%\item damping factor $\zeta = 0.8$
				%\begin{enumerate}
					%\item Using these values, feedback gains $m_1$ and $m_2$ were evaluated to be 
					%\item $m_1=42.45$, $m_2=771.13$ 
				%\end{enumerate}
		%\item external disturbance $d_{ext}= 200$ rad/$s^2$
		%\item desired roll orientation $=0$ deg
		%\item initial condition in $\phi= 10$ deg
		%\item all other values as referred from Table.1
\end{itemize}
\end{frame}
%where , respectively. Now simulations were carried out using the control law (\ref{eq11})  for the simulation parameters discussed before. It may be noted that the effects of actuator dynamics are not accounted in the UDE based control law. The simulation results are presented in Fig. 3.
%-------------------------------------------------------------------------------Figure 4-------------------------------------------------------------------
%
\begin{frame} 
\frametitle{Comparative analysis (Case 2)}
%\textbf{Simulation Output}
%\begin{figure}[h]
%\begin{center}
	%%\begin{subfigure}
	%%\subfigure[Output Response]{\includegraphics[width=8.4cm]{fig2a}}
	%{\includegraphics[width=4cm]{fig4a}}
	%%\subfigure[Output Response]{\includegraphics[width=4cm]{fig4a}}
	%%\includegraphics[width=8.4cm]{fig3a}    % The printed column width is 8.4 cm.
	%%\caption{Output response}
	%%\end{subfigure}
%%
	%%\begin{subfigure}
	%%\subfigure[Control Effort]{\includegraphics[width=4cm]{fig4b}}
	%%\includegraphics[width=8.4cm]{fig3b}    % The printed column width is 8.4 cm.
	%%\caption{Control Effort}
	%%\end{subfigure}
	%%\subfigure[Disturbance Estimation]{\includegraphics[width=4cm]{fig4c}}
	%\caption{Performance with UDE with second order actuator} 
%\label{fig4}
%\end{center}
%\end{figure}
%\includegraphics[width=3cm]{fig4a}
%%\caption{Output response}
%\includegraphics[width=3cm]{fig4b}
%\includegraphics[width=3cm]{fig4c}

\begin{figure}
\includegraphics[width=3.7cm]{control_kcn_13_kcm_07}
%\title{Output response}
\includegraphics[width=3.7cm]{pitch_kcn_13_kcm_07}
%\caption{Output response}
%\end{figure}
\begin{center}
\includegraphics[width=3.7cm]{x1_kcn_13_kcm_07}
\end{center}
\end{figure}
\end{frame}

\begin{frame}
\frametitle{Comparative analysis (Case 2)}
	ADD SOME CONTENT HERE FOR CASE 2
\end{frame}


\begin{frame}
	\frametitle{Results and conclusions}
\end{frame}

\begin{frame}
	\frametitle{Novelty and future work}
\end{frame}

\begin{frame}
	\frametitle{References}
\end{frame}
%---------------------------------------------------------------------------------------------------------------------------------


	\begin{thebibliography}{1}



%\bibitem{IEEEhowto:kopka}
%H.~Kopka and P.~W. Daly, \emph{A Guide to \LaTeX}, 3rd~ed.\hskip 1em plus
  %0.5em minus 0.4em\relax Harlow, England: Addison-Wesley, 1999.
	%
%% Reference 1
%\bibitem{song2002}
%Chanho Sung, and Yoon-Sik Kim, \lq \lq A new approach to motion modeling and autopilot design of skid-to-turn missile," Trans. on Control, Automation, and System Engg, 4(3), Sep 2002, pp. 231-238. 
%
%% Reference 2
%\bibitem{kang2009}
%S. Kang, and H. J. Kim, \lq \lq Roll-pitch-yaw integrated robust autopilot design for a high angle-of-attack missile," J. of Guidance, Control, and Dynamics, Sep-Oct 2009, pp. 1622-1628.
%
%%Reference 3
%\bibitem{sirisha2012}
%C. V. Sirisha, Ranajit Das, and R. N. Bhattacharjee, \lq \lq Disturbance estimation based roll autopilot design for tactical missiles," Proc. Advances in Control and Optimization of Dynamic Systems, ACDOS - 2012, pp. 1-5.



%Reference 13
%\bibitem{zhong2004}
%Q. C. Zhong, and D. Rees, \lq \lq Control of LTI systems based %on an uncertainty and disturbance estimator," ASME Trans. J. of %Dynamic systems, Measurement and Control, 126(4), 2004, pp. %905-910.

%Reference 14
%\bibitem{ogata2010}
%K. Ogata, Modern Control Engineering, 5th ed. PHI, New Delhi, %2010, pp. 743-746.
\bibitem{b1} Y. Dong, J. Li. and T. Li, ``Generalized extended state observer based H$\infty$ control design for a missile longitudinal autopilot," J. of Aerosp. Eng., vol. 230, no. 12, pp. 2162–-2178, 2015. \label{geso}

\bibitem{b2} A. A. Godbole, T. R. Libin and S. E. Talole, ``Extended state observer-based robust pitch autopilot design for tactical missiles," J. of Aerosp. Eng., vol. 226, no. 12, pp. 1482–-1501, 2011. \label{eso}

\bibitem{b3}  Q.-C. Zhong and D. Rees, ``Control of uncertain LTI systems based on an uncertainty and disturbance estimator," Journal of Dyn. Sys. Meas. Con.Trans. ASME, vol. 126, no. 4, pp. 905–-910, 2004. \label{ude1}

\bibitem{b4} S.E. Talole, T.S. Chandar and J. P. Kolhe, ``Design and experimental validation of UDE based controller--observer structure for robust input--output linearisation," Int. J. of Control, vol. 84, no. 5, pp. 969-984, 2011. \label{talole2011}


\end{thebibliography}
%--------------------------------------------------------------------------
\subsection*{Thank You} % A subsection can be created just before a set of slides with a common theme to further break down your presentation into chunks
\begin{frame}
%\hskip4cm
\textbf{\LARGE Thank You!} \\[20pt]
%\hskip3cm
\end{frame}

\end{document} 
