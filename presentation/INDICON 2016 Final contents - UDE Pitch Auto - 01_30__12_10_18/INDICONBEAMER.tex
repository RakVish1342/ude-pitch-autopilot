%%%%%%%%%%%%%%%%%%%%%%%%%%%%%%%%%%%%%%%%%
% Beamer Presentation
% Standard LaTeX Template used for creating presentation of Firebird-V Robot and other tutorials. 
% Author: Saurav Shandilya (e-Yantra Team)
% Reference: www.LaTeXTemplates.com Version 1.0 (10/11/12)
%
%%%%%%%%%%%%%%%%%%%%%%%%%%%%%%%%%%%%%%%%%

%----------------------------------------------------------------------------------------
%	PACKAGES AND THEMES
%----------------------------------------------------------------------------------------
		
\documentclass[table,10pt,red]{beamer}	% First line -- Define document class as Beamer which is used for creating presentation using Latex
\setbeamercolor{alerted text}{fg=blue} 	% Sets color of highlighted text during presentation.  
 

% The Beamer class comes with a number of default slide themes
% which change the colors and layouts of slides. Below this is a list
% of all the themes, uncomment each in turn to see what they look like.

\usetheme{default} 
%\usetheme{AnnArbor} %N
%\usetheme{Antibes} %M
%\usetheme{Bergen} %N
%\usetheme{Berkeley} %N
%\usetheme{Berlin}		%used theme in present documents.
%\usetheme{Boadilla} %N
%\usetheme{CambridgeUS} %N
%\usetheme{Copenhagen} %N
%\usetheme{Darmstadt} %N
%\usetheme{Dresden} %Y
%\usetheme{Frankfurt} %N
%\usetheme{Goettingen} %N
%\usetheme{Hannover} %N
%\usetheme{Ilmenau} %N
%\usetheme{JuanLesPins} %Y
%\usetheme{Luebeck} %Y
%\usetheme{Madrid} %N
%\usetheme{Malmoe} %Y
%\usetheme{Marburg} %N
%\usetheme{Montpellier} %Y
%\usetheme{PaloAlto} %N
%\usetheme{Pittsburgh} %Y
%\usetheme{Rochester} %Y
%\usetheme{Singapore} %yes
%\usetheme{Szeged}
%\usetheme{Warsaw}

% As well as themes, the Beamer class has a number of color themes
% for any slide theme. Uncomment each of these in turn to see how it
% changes the colors of your current slide theme.
 
%%Szeged+dove

%\usecolortheme{albatross} %dark blue N
%\usecolortheme{beaver} %white and grey
%\usecolortheme{beetle} %blue and dark grey N
%\usecolortheme{crane} %yellow blue base: white
%\usecolortheme{dolphin} %pink N
%\usecolortheme{dove} %black and white
%\usecolortheme{fly} base: grey N
%\usecolortheme{lily}  %red white black (previously used)
%\usecolortheme{orchid} %similar to lily N
%\usecolortheme{rose}  %similar to lily N
%\usecolortheme{seagull} %grey and white. N
%\usecolortheme{seahorse}% red pink base white N
%\usecolortheme{whale} similar to lily N
%\usecolortheme{wolverine} %yellow red orange base white

%\setbeamertemplate{footline} % To remove the footer line in all slides uncomment this line
%\setbeamertemplate{footline}[page number] % To replace the footer line in all slides with a simple slide count uncomment this line

%\setbeamertemplate{navigation symbols}{} % To remove the navigation symbols from the bottom of all slides uncomment this line
%}

%------------------------------------------------------------------------------------------
%	\usepackage is required for including various features like images, table, references etc.
%	Packages must be installed before using. These can be istalled through package manager. 
%   Various packages have dependencies and for using such packages all dependent packages must be used. 
%-----------------------------------------------------------------------------------------
%\usepackage{beamerthemeshadow} % theme shadow for visual 
\usepackage{beamerthemesplit} % Creates minipage (for showing multiple images and text) on same page  
\usepackage{graphicx} % Allows including images
\usepackage{booktabs} % Allows the use of \toprule, \midrule and \bottomrule in tables
\usepackage{xcolor}
\usepackage{booktabs,array}
\usepackage{listings}
\usepackage{hyperref}	% Required for including hyperlink in document
\usepackage{verbatim,moreverb} % Required for including code snippet.
\usepackage{colortbl}
\usepackage{multirow}	% Required for creating multiple row tables
\usepackage{tikz}		% Required for drawing shapes such as circles, arrowed line, etc. 
\usetikzlibrary{arrows}

% logo
%\logo{\includegraphics[height=1cm]{iitblogo.pdf}} % includes logo at bottom of all slides 

%----------------------------------------------------------------------------------------
%	TITLE PAGE
%----------------------------------------------------------------------------------------
% sf family, bold font
\sffamily \bfseries
% content inside [] appears at bottom of all page. content inside {} appears on first page as title. double backslash means line change 
\title
[
Uncertainty and Disturbance Estimator Based Robust Pitch Autopilot	% bottom of all page
%	\hspace{0.5cm}
%	\insertframenumber/\inserttotalframenumber
]
{
	Uncertainty and Disturbance Estimator Based Robust Pitch Autopilot
}

\author
[
%RAKSHITH VISHWANATHA, SHARATH RAO, ABHISHEK BASRITHAYA, T.S. CHANDAR	%Name at bottom of all page 
]
% author name on title slide
{
	RAKSHITH VISHWANATHA, SHARATH RAO, \\ABHISHEK BASRITHAYA, T.S. CHANDAR \\
  %P.E.S. Institute of Technology,Bangalore\\
}
%\date
%{
%P.E.S. Institute of Technology, Bangalore, India \\ {\today}	%\today picks system date on title slide
%}
\date
{
P.E.S. Institute of Technology, Bangalore, India \\ 17th October 2018	%\today picks system date on title slide
}
\begin{document} % IN LATEX ALL DOCUMENT/REPORT/PRESENTATION STARTS WITH \begin{document} AND ENDS WITH \end{document}

\begin{frame}	% FRAME MEANS SLIDE. \begin{frame} STARTS THE SLIDE AND \end{frame} ENDS THE SLIDE
	\titlepage % Print the title page as the first slide
\end{frame}
\section*{Outline}
% START OF SECOND SLIDE
\begin{frame}
	\frametitle{Outline of work} % Table of contents slide, comment this block out to remove it
	\tableofcontents % Throughout your presentation, if you choose to use \section{} and \subsection{} commands, these will automatically be printed on this slide as an overview of your presentation
\end{frame}
%----------------------------------------------------------------------------------------
%	PRESENTATION SLIDES
%----------------------------------------------------------------------------------------

\section{Problem Statement} 
\begin{frame}
\frametitle{Problem Statement}
%\textbf{How do actuators pose a problem in system stability?}
\begin{itemize}  % Shows text in bullet point 	
	\item Missile control systems have traditionally faced many challenges in predicting the uncertainty and disturbance acting on the missile at various points in its flight envelope. %Actuators cause drastic deterioration in stability unless compensation is provided 
	\medskip
	\item Few robust control techniques (SMC, PC) can tackle non-linear system, but it requires prior knowledge of the magnitude bounds of the disturbance acting on the system %One of the reasons for this instability, is lag created by the actuators
	\medskip
	\item Uncertainty and Disturbance Estimator (UDE) has been employed to address these concerns
\end{itemize}
\end{frame}
%------------------------------------------------

%------------------------------------------------SECTION1-----------------------------------------------------------------------------------
\section{Missile Model} % Sections can be created in order to organize your 
\subsection{Governing Equations}
	\begin{frame}
	\frametitle{Missile Model}
	 
	%\textbf{Block diagram schematic for roll dynamics of a tactical missile} 	
 	\begin{itemize}  % Shows text in bullet point 		
		\item The missile model is a pitch axis,
		longitudinal, tail controlled missile with nonlinear dynamics [\ref{geso}];
		\begin{enumerate}
				\item $\dot{\alpha}(t)=	K_\alpha M(t) C_n [\alpha(t),\delta(t),M(t)]cos(\alpha(t)) + q(t) $
				\item $\dot{q}(t) =	K_q M^2(t) C_m [\alpha(t),\delta(t),M(t)]$\\
				\item $\dot{\delta}(t)	=	-\omega_a\delta(t) + \omega_a\delta_c(t)$
 		\end{enumerate}
		\end{itemize}
	\begin{itemize}
		\item  where;
			\begin{enumerate}
				\item $C_n(\alpha,\delta,M)=a_n\alpha^3+b_n\alpha|\alpha|+c_n\Big(2-\frac{M}{3}\Big)\alpha\quad+ d_n\delta$
				%\item Robots and external devices
				\item $C_m(\alpha,\delta,M)=a_m\alpha^3+b_m\alpha|\alpha|+ c_m\Big(-7+\frac{8M}{3}\Big)\alpha 
				\quad+d_m\delta$
					
				 \item $\dot{M}(t)=\frac{1}{\nu_s}[-|a_z(t)|sin|\alpha(t)|+a_xM^2(t)cos\alpha(t)]$
				 %\item $a_z = K_z M^2(t)C_n[\alpha(t),\delta(t),M(t)]$
				 %\item $a_x = \frac{0.7P_0SC_d}{m}$
			\end{enumerate}
	\end{itemize}
	\begin{itemize}
		\item Assumes no roll rate, no sideslip and no yaw rate
		\item In this model $\delta_c$ is the commanded input to the tail fins and $\alpha$ is the angle of attack
\end{itemize}

	%---------------------------------------------------figure 1 block diagram schematic------------------------------------------------------

	\end{frame}
 	%----------------------------next slide------------------------------------------------------------------------------------------
 	
 	\begin{frame}
 	\frametitle{Aerodynamic constants}
 	\begin{itemize}
 		\item Model design constraints:
 		\begin{enumerate}
 			\item $-20^\circ\le \alpha \le20^\circ$
 			\item $1.5 \le M \le 3$
 			\item $\pm25\%$ uncertainty in $C_n$ and $C_m$
 		\end{enumerate}
 	\end{itemize}
 	\begin{table}[h]
 		\begin{center}
 			\caption{Performance specifications}\label{tb1}
 			\begin{tabular}{ccc}
 				\hline
 				$\omega_a$ & Actuator bandwidth & 150 rad/s\\ \hline
 				$\zeta_a$ & Actuator damping & 0.7 \\ \hline
 				$m$ & Mass & 204.023 kg\\ \hline
 				$d$ & Diameter & 0.2286 m \\ \hline
 				$I_y$ & Pitch moment of inertia & 247.44 $kgm^2$\\ \hline
 				$C_d$ & Drag coefficient & 0.3\\ \hline
  				$v_s$ & Speed of sound & 315.89 $m/s$ \\ \hline
 				$M$ & Mach & 2.25 \\ \hline
 			\end{tabular}
 		\end{center}
 	\end{table}
 \end{frame}

\subsection{Overview of UDE Controller-Observer} 	
\begin{frame}
	\frametitle{Overview of UDE Controller-Observer}
		\begin{figure}
			\includegraphics[width=0.8\linewidth]{block_diag}
		\end{figure}

\end{frame}


%------------------------------------------------
%SUBSECTION1
\subsection{Controller and Observer Design} % A subsection can be created just before a set of slides with a common theme to further break down your presentation into chunks
\begin{frame}
	\frametitle{Input Output Linearization}
	\textbf{Implementation of IOL to cancel non-linearities in model [\ref{eso}]} 
	\begin{itemize}
		\item To make relative degree = order of equation, $d_n \approx 0$
		\item Within $-20^\circ\le \alpha \le20^\circ$, $\cos(\alpha) \approx 1$
		\item Now obtaining $\dddot{\alpha}$ gives
	\end{itemize}

\begin{eqnarray*}
\begin{aligned}
	\dddot{\alpha}&=K_q M^2(3a_m\alpha^2+2b_m|\alpha|+c_m\Big(-7+\frac{8M}{3}\Big))\dot{\alpha}\\ 
	&\quad - K_q M^2d_m\omega_a\delta+K_{\alpha}M(6a_n\alpha+2b_n sgn(\alpha))\dot{\alpha}^2\\ 
	&\quad + K_\alpha M(3a_n\alpha^2+2b_n|\alpha|+c_n\Big(2-\frac{M}{3}\Big))\ddot{\alpha}\\ 
	&\quad + K_q M^2d_m\omega_a\delta_c \label{a3dot}
\end{aligned}
\label{eq4}
\end{eqnarray*}

\end{frame}

\begin{frame}
\frametitle{Input Output Linearization}
When represented in IOL form $\dddot{\alpha} = a + b\delta_c$ we get; 
\begin{eqnarray*}
	\begin{aligned}
		a &= K_qM^2(3a_m\alpha^2+2b_m|\alpha|+c_m\Big(-7+\frac{8M}{3}\Big))\dot{\alpha} \\ 
		&\quad -K_qM^2d_m\omega_a\delta+K_\alpha M(6a_n\alpha+2b_nsgn(\alpha))\dot{\alpha}^2\\ 
		&\quad+K_\alpha M(3a_n\alpha^2+2b_n|\alpha|+c_n\Big(2-\frac{M}{3}\Big))\ddot{\alpha} \\
		b &= K_q M^2d_m\omega_a \nonumber
	\end{aligned}
	\label{eq5}
\end{eqnarray*}

Thus the control law is, 

\begin{eqnarray*}
	\begin{aligned}
%		without this comment the equations on this slide disappear. No idea why.
		\delta_c &= \frac{1}{b}(u_a+\nu) \\ \label{iol_control}
		u_a &= -a \label{ua_eqn}\\
		\nu &= \dddot{\alpha}^\ast+m_1(\alpha^\ast-\alpha) + m_2(\dot{\alpha}^\ast-\dot{\alpha}) + m_3(\ddot{\alpha}^\ast-\ddot{\alpha}) \label{nu_eqn}
	\end{aligned}
	\label{eq5}
\end{eqnarray*}

\end{frame}
%-------------------------------------------------------------------------------------------------------------------------------------------
\begin{frame}
\frametitle{UDE Augmented IOL Controller}
\textbf{The UDE control law utilizes a new term $u_d$ as follows [\ref{ude1}];}
\begin{itemize}  % Shows text in bullet point 		
		\item $\delta_c = \frac{1}{b}\Big[u_a+u_d+\nu\Big]$ where, $u_d = -\hat{d}$
		\item $\hat{d}$ is an estimate of the lumped disturbance and uncertainties $d$;
%\end{itemize}

%\textbf{$\hat{d}$ is an estimate of the lumped disturbance and uncertainties $d$;}
%\begin{itemize}
		\begin{enumerate}
			\item $\dddot{\alpha} = a + b\delta_c + d$
			\item $d = \Delta a + \Delta b \delta_c + w$
			\item $\hat{d}=G_f(s)d$
			\item $G_f(s)=\frac{1}{1+s\tau}$ \\
		\end{enumerate}
\end{itemize}

\begin{itemize}
	\item Thus, we finally get
	$u_d=\frac{-1}{\tau}\Big[\ddot{\alpha}-\int{\nu dt}\Big] \label{ude}$
	
	\item UDE does not require knowledge of magnitude bounds and only requires the frequency bounds (tuned using $\tau$) of the uncertainty or disturbance
\end{itemize}

\end{frame}
%%------------------------------------------------------------------------------------------------------------------------------------------
\begin{frame}
\frametitle{UDE Observer based Control law}
	\textbf{A Luenberger like UDE Observer has been designed [\ref{talole2011}] }


\begin{itemize}  % Shows text in bullet point 		
	\item $\dddot{\alpha}$ is separated into linear and non-linear ($d_1$) terms; $\dddot{\alpha} = a_1 \alpha + a_2 \dot{\alpha} + a_3 \ddot{\alpha} + d_1 + b\delta_c$
	
	\item Then the non-linear term $d_1$ and uncertainties are clubbed into a lumped term $d_2$, such that; \\
	$\dddot{\alpha} = a_{1o} \alpha + a_{2o} \dot{\alpha} + a_{3o} \ddot{\alpha} + b_o\delta_c + d_2$\\
	
	\item This expressed in state space form can then be used to design the observer:
	\begin{eqnarray*}
		\begin{aligned}
			\dot{x}_1 &= x_2 \\
			\dot{x}_2 &= x_3 \\
			\dot{x}_3 &= a_{1o}x_1 + a_{2o}x_2 + a_{3o}x_3 + b_o \delta_c + d_2 \\
			y &= x_1 \label{rx1}
		\end{aligned}
		\label{eq5}
	\end{eqnarray*}
	
\end{itemize}

\end{frame}

\begin{frame}
\frametitle{UDE Observer based Control law}

Now since the equations are represented in a linear manner, a Luenburger like UDE Observer is designed by introducing the observer poles $[\beta_1 \beta_2 \beta_3]$
\begin{eqnarray*}
	\begin{aligned}
		\dot{\hat{x}}_1 &= \hat{x}_2 + \beta_1 e_o\\
		\dot{\hat{x}}_2 &= \hat{x}_3 + \beta_2 e_o\\
		\dot{\hat{x}}_3 &= a_{1o}\hat{x}_1 + a_{2o}\hat{x}_2 + a_{3o}\hat{x}_3 + b \delta_c + \hat{d}_2 + \beta_3 e_o\\		
		\hat{y} &= \hat{x}_1 \label{ss1}
	\end{aligned}
	\label{eq5}
\end{eqnarray*}

Here, the term $d_2$ representing the non-linearities and uncertainties is estimated by UDE  as $\hat{d_2}$

\end{frame}

\begin{frame}
\frametitle{Overview of UDE Controller-Observer}
\begin{figure}
	\includegraphics[width=0.8\linewidth]{block_diag}
\end{figure}

\end{frame}


\subsection{Stability Analysis}
\begin{frame}
\frametitle{Stability Analysis}
\textbf{Error dynamics for UDE missile autopilot controller-observer structure}
\begin{eqnarray}
\begin{aligned}
\begin{bmatrix}
\dot{e}_c \\
\dot{e}_o \\
\dot{\tilde{d}}_2
\end{bmatrix} =& 
\begin{bmatrix}
(A - BK) & -(BK) & -B_d \\
0 & (A - LC) & B_d \\
0 & 0 & -\frac{1}{\tau}
\end{bmatrix}
\begin{bmatrix}
e_c \\
e_o \\
\tilde{d}_2
\end{bmatrix}	
& + 
\begin{bmatrix}
0 \\
0 \\
1
\end{bmatrix} \dot{d}_2 \label{sr8}
\end{aligned}
\end{eqnarray}
Eigen values of the system matrix can be computed from
\begin{eqnarray}
\begin{vmatrix}
sI - (A - BK)
\end{vmatrix}
\begin{vmatrix}
sI - (A - LC)
\end{vmatrix}
\begin{vmatrix}
s - (-\frac{1}{\tau})
\end{vmatrix} = 0 \label{sr9}
\end{eqnarray}

\begin{itemize}
	\item $(A, B)$ is controllable and $(A, C)$ is observable
	\item $\tau$ is strictly a positive number
	\item Selecting appropriate controller and observer poles ensures stability of error dynamics
	\item Also if $\dot{d}_2 \ne 0$, then bounded-input, bounded-output stability can be assured. 
\end{itemize}


\end{frame}


%%--------------------------------------------- --
%% Start of fifth slide
%%------------------------------------------------------------------------------------------------------------------------------------------
\section{Simulations}
\begin{frame}
\frametitle{Simulations}
\textbf{Parameters for simulation}
\begin{itemize}  % Shows text in bullet point 		
		\item Reference signal: \\  
		\begin{equation}
			\alpha^{*}=
			\begin{cases}
			15^{\circ}, & \text{if $0 \leq t \leq2 ~ s$}\\
			-8^{\circ}, & \text{if $2 < t \leq4 ~ s$}\\
			10^{\circ}, & \text{if $4 < t \leq6 ~ s$} \label{ref_sig}\\ 
			\end{cases} \nonumber
		\end{equation}
		\item Tracking Constraints:
		\begin{enumerate}
			\item To be tracked with a time constant of less than $0.25s$
			\item Less than 10\% overshoot
			\item Less than 1\% steady-state error
		\end{enumerate}
		\item Pole placement to meet this target:
		\begin{enumerate}
			\item Controller gains $[m_1\ m_2 \ m_3]$ placed at $s_{1,2,3} = -12$
			\item Observer gains $[\beta_1 \ \beta_2 \ \beta_3]^T$ placed at $s_{1,2,3} = -360$
		\end{enumerate}
	
		\item Controller and observer designed at M = 2.25 (mid-point of Mach envelope)
	\end{itemize}

\end{frame}

%%-----------------------------------------------------------------------------------------------------------------------------------------
\subsection{Case I: UDE with Mach Dynamics \& External Disturbance}
\begin{frame}
\frametitle{Case I: UDE with Mach Dynamics \& External Disturbance}

\begin{itemize}
	\item UDE simulated with varying mach and external disturbance
	\item Mach i.c is $M(0) = 2.5$ and follows $M$ equation [\ref{geso}] till $M = 2$
	\item External disturbance modeled as sinusoidal amplitude $8^\circ$ and frequency $0.25 Hz$
\end{itemize}

\begin{figure}
\includegraphics[width=4cm]{1_ude_varying-mach_x1}
%\title{Output response}
\includegraphics[width=4cm]{2_ude_varying-mach_control}
\end{figure}


\end{frame}	

\begin{frame}
\frametitle{Case I: UDE with Mach Dynamics \& External Disturbance}
\begin{figure}
	\includegraphics[width=4cm]{3_ude_varying-mach_mach}
	%\title{Output response}
	\includegraphics[width=4cm]{4_ude_varying-mach_dist}
\end{figure}
	\begin{center}
	\includegraphics[width=4cm]{5_ude_varying-mach_pitch}
	\end{center}
\end{frame}

\subsection{Case II: Comparative Study With Aerodynamic Uncertainties}
\begin{frame}
\frametitle{Case II: Comparative Study With Aerodynamic Uncertainties}
\begin{itemize}  % Shows text in bullet point 		
		\item Comparative analysis of UDE has been done against Predictive Control (PC) [\ref{talole2011}] and Sliding Mode Control (SMC) [\ref{bahrami}]
		\bigskip
		\item Uncertainty of $+30\%$ in aerodynamic force coefficient $C_n$ and $-30\%$ in aerodynamic moment coefficient $C_m$
		\bigskip
		\item Mach has been maintained at the nominal constant of $M = 2.25$ 
		\bigskip
		\item No external disturbances added to the system.
\end{itemize}
\end{frame}
%-------------------------------------------------------------------------------Figure 4-------------------------------------------------------------------
%
\begin{frame} 
\frametitle{Case II: Comparative Study With Aerodynamic Uncertainties}
\begin{figure}
\includegraphics[width=4cm]{x1_kcn_13_kcm_07}
%\title{Output response}
\includegraphics[width=4cm]{control_kcn_13_kcm_07}
%\caption{Output response}
%\end{figure}
\begin{center}
\includegraphics[width=4cm]{pitch_kcn_13_kcm_07}
\end{center}
\end{figure}
\end{frame}

\section{Results and Conclusion}
\begin{frame}
	\frametitle{Results and conclusions}
	\begin{itemize}
		\item Results of CASE I
		\begin{enumerate}
			\item Tracking performance is as desired and control effort stays smooth and within the practical bounds of  $\pm30^\circ$
			\item UDE observer is able to estimate states quickly and accurately
			\item Estimation of disturbance by UDE is minimally delayed and follows closely to the actual value
		\end{enumerate}
		
		\item Results of CASE II
		\begin{enumerate}
			\item Due to aerodynamic uncertainty, PC has overshoots in tracking. UDE and SMC are smooth 
			\item Control effort of PC is oscillatory. SMC has high overshoots at transition points. UDE is able to provide smooth control within $\pm 30^\circ$
			\item Pitch graph of PC is oscillatory while SMC is slightly delayed. Pitch graph of UDE is on point.
		\end{enumerate}
	\end{itemize}	
	
\end{frame}

\section{Novelty and Future Work}
\begin{frame}
	\frametitle{Novelty}
	\begin{itemize}
		\item Despite uncertainties such as varying Mach (case I) and varying aerodynamic constants (case II) and even the presence of external disturbance (case I), UDE is able to provide robust tracking control.
		\bigskip
		
		\item Only the frequency bound (captured through $\tau$) of the external disturbance and uncertainty is required to provide robust tracking with UDE. There is no dependency on the magnitude bounds of the disturbance/uncertainty.
		\bigskip
		
		\item Unlike PC and SMC simulations which have used the actual states while implementing the control law, UDE simulation has utilized the estimated
		states obtained from the UDE observer. As is well known, use of estimated states might result in degraded performance; in contrast the proposed strategy utilizes the estimated states and still proves its worthiness.
\end{itemize}
\end{frame}


\begin{frame}
\frametitle{Future Work}
Our future work would include the design of an integrated pith-yaw-roll autopilot for a nonlinear missile model with and without linearization, using UDE theory.

\end{frame}


\begin{thebibliography}{1}

	\bibitem{b1} Y. Dong, J. Li. and T. Li, ``Generalized extended state observer based H$\infty$ control design for a missile longitudinal autopilot," J. of Aerosp. Eng., vol. 230, no. 12, pp. 2162–-2178, 2015. \label{geso}
	
	\bibitem{b2} A. A. Godbole, T. R. Libin and S. E. Talole, ``Extended state observer-based robust pitch autopilot design for tactical missiles," J. of Aerosp. Eng., vol. 226, no. 12, pp. 1482–-1501, 2011. \label{eso}
	
	\bibitem{b3}  Q.-C. Zhong and D. Rees, ``Control of uncertain LTI systems based on an uncertainty and disturbance estimator," Journal of Dyn. Sys. Meas. Con.Trans. ASME, vol. 126, no. 4, pp. 905–-910, 2004. \label{ude1}
	
	\bibitem{b4} S.E. Talole, T.S. Chandar and J. P. Kolhe, ``Design and experimental validation of UDE based controller--observer structure for robust input--output linearisation," Int. J. of Control, vol. 84, no. 5, pp. 969-984, 2011. \label{talole2011}
	
	\bibitem{b5} M. Bahrami, B. Ebrahimi, G. R. Ansarifar and J. Roshanian, ``Sliding mode autopilot and observer design for a supersonic flight vehicle," in 2nd Int. Symp. on Syst. and Control in Aerosp. and Astronautics, Shenzhen, pp. 1-5, 2008.\label{bahrami}
	
\end{thebibliography}%--------------------------------------------------------------------------
\subsection*{Thank You} % A subsection can be created just before a set of slides with a common theme to further break down your presentation into chunks
\begin{frame}
%\hskip4cm
\textbf{\LARGE Thank You!} \\[20pt]
%\hskip3cm
\end{frame}

\end{document} 
